  % ------------------------------------------------
  % Minimal KDD Explorations Report Template
  % Based on sigkddExp.cls (https://www.kdd.org/author-instructions)
  % ------------------------------------------------

  \documentclass{sigkddExp}

  % --- Title & Author Info ---
  \title{Scalable Local-Recoding Anonymization using Locality
Sensitive Hashing for Big Data Privacy Preservation}

  \numberofauthors{4}

  \author{
    \alignauthor Khotso Bore\\
      \affaddr{University Of Pretoria}\\
      \email{u19180642@tuks.co.za}
    \alignauthor Innocentia Ledimo\\
      \affaddr{University Of Pretoria}\\
      \email{u22928678@tuks.co.za}
    \and
    \alignauthor Busisiwe Vemba\\
      \affaddr{University Of Pretoria}\\
      \email{u22928678@tuks.co.za}
    \alignauthor Siphesihle Khumalo\\
      \affaddr{University Of Pretoria}\\
      \email{u25759257@tuks.co.za}
    }

  \date{\today}

\begin{document}
\maketitle

\begin{abstract}
  A brief summary of your project.
  Mention the dataset, goal (e.g., clustering/classification/EDA), and key findings in 4–5 sentences.
\end{abstract}

% ------------------------------------------------
\section{Exploratory Data Analysis (EDA)}

\subsection{Data Inspection}
Describe your dataset: number of rows, columns, types of variables, missing values, and summary statistics.
Include a short table or paragraph summarizing key properties.

\begin{table}[h]
  \centering
  \caption{Dataset Overview}
  \begin{tabular}{lcc}
    \hline
    Feature & Type        & Missing (\%) \\
    \hline
    Age     & Numerical   & 2.3          \\
    Gender  & Categorical & 0.0          \\
    Income  & Numerical   & 5.1          \\
    \hline
  \end{tabular}
\end{table}

\subsection{Visualisations}

Here are some key visualizations from the EDA phase. We explore the distributions of the numwerical features age and hours worked, and their correlations.

\begin{figure}[h]
  \centering
  \includegraphics[width=0.75\linewidth]{images/age_distribution_2.png}
  \caption{Age Distribution}
\end{figure}

\begin{figure}[h]
  \centering
  \includegraphics[width=0.75\linewidth]{images/hours_per_week_distribution_2.png}
  \caption{Hours Per Week Distribution}
\end{figure}

\begin{figure}[h]
  \centering
  \includegraphics[width=0.75\linewidth]{images/age_hours_correlation_matrix.png}
  \caption{Correlation Matrix of Numerical Features}
\end{figure}

\pagebreak
\subsection{Insights}
% Summarize interesting patterns:
% \begin{itemize}
%   \item Which variables correlate strongly?
%   \item Any skewed distributions or outliers?
%   \item Early hypotheses about clusters or classes
% \end{itemize}

The correlation between age and hours worked suggests that while there may be slight tendencies—such as younger or older employees working somewhat more or fewer hours—the relationship is not strong or consistent across the group.

A large portion of employees work standard full-time hours (around 40 hours per week), with fewer employees working significantly more or less than this amount. Very likely because most records are from the United States, with a smaller representation from other countries.

% ------------------------------------------------
\section{Data Preprocessing}

\subsection{Handling Missing Data}
Explain how you dealt with missing data (imputation, deletion, etc.) and justify your choice.

\subsection{Feature Engineering}
List any new variables or transformations you applied (e.g., encoding, log transforms, ratios).

\subsection{Standardisation / Normalisation}
Discuss any scaling applied (e.g., z-score, min–max) and why it was necessary.

% ------------------------------------------------
\section{Data Mining Methods and Analysis}

\subsection{Methods}
Describe which algorithms or analytical methods were applied:
\begin{itemize}
  \item Clustering (e.g., K-Means, DBSCAN)
  \item Dimensionality Reduction (e.g., PCA, t-SNE)
  \item Classification/Regression (if applicable)
\end{itemize}

\subsection{Results}
Summarize the main results. Use figures/tables for clarity:
\begin{itemize}
  \item Cluster quality metrics (e.g., silhouette score)
  \item Feature importances
  \item Visualizations of clusters or decision boundaries
\end{itemize}

\subsection{Discussion}
Interpret the results:
\begin{itemize}
  \item What patterns or groups emerged?
  \item Were the methods appropriate?
  \item Any limitations or anomalies?
\end{itemize}

% ------------------------------------------------
\section{Conclusion and Reflection}
Summarize what you found and learned:
\begin{itemize}
  \item Key insights from data mining
  \item Challenges faced and how you overcame them
  \item Potential future work or improvements
\end{itemize}

% ------------------------------------------------
\section*{Acknowledgements}
 (Optional) Acknowledge any data sources, collaborators, or funding.

% ------------------------------------------------
\bibliographystyle{abbrv}
\bibliography{references}

\end{document}
