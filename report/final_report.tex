% ------------------------------------------------
% Minimal KDD Explorations Report Template
% Based on sigkddExp.cls (https://www.kdd.org/author-instructions)
% ------------------------------------------------

\documentclass{sigkddExp}

% --- Title & Author Info ---
\title{Project Title: A Short, Descriptive Title}

\numberofauthors{4}

\author{
\alignauthor First Author\\
  \affaddr{University / Organization}\\
  \email{first.author@email.com}
\alignauthor Second Author\\
  \affaddr{University / Organization}\\
  \email{second.author@email.com}
}

\date{\today}

\begin{document}
\maketitle

\begin{abstract}
A brief summary of your project.  
Mention the dataset, goal (e.g., clustering/classification/EDA), and key findings in 4–5 sentences.
\end{abstract}

% ------------------------------------------------
\section{Exploratory Data Analysis (EDA)}

\subsection{Data Inspection}
Describe your dataset: number of rows, columns, types of variables, missing values, and summary statistics.  
Include a short table or paragraph summarizing key properties.

\begin{table}[h]
\centering
\caption{Dataset Overview}
\begin{tabular}{lcc}
\hline
Feature & Type & Missing (\%) \\
\hline
Age & Numerical & 2.3 \\
Gender & Categorical & 0.0 \\
Income & Numerical & 5.1 \\
\hline
\end{tabular}
\end{table}

\subsection{Visualisations}
Include key plots to explore distributions or relationships:
\begin{itemize}
  \item Histogram of key numerical features
  \item Boxplot to show outliers
  \item Pairplot / Correlation heatmap for relationships
\end{itemize}

\subsection{Insights}
Summarize interesting patterns:
\begin{itemize}
  \item Which variables correlate strongly?
  \item Any skewed distributions or outliers?
  \item Early hypotheses about clusters or classes
\end{itemize}

% ------------------------------------------------
\section{Data Preprocessing}

\subsection{Handling Missing Data}
Explain how you dealt with missing data (imputation, deletion, etc.) and justify your choice.

\subsection{Feature Engineering}
List any new variables or transformations you applied (e.g., encoding, log transforms, ratios).

\subsection{Standardisation / Normalisation}
Discuss any scaling applied (e.g., z-score, min–max) and why it was necessary.

% ------------------------------------------------
\section{Data Mining Methods and Analysis}

\subsection{Methods}
Describe which algorithms or analytical methods were applied:
\begin{itemize}
  \item Clustering (e.g., K-Means, DBSCAN)
  \item Dimensionality Reduction (e.g., PCA, t-SNE)
  \item Classification/Regression (if applicable)
\end{itemize}

\subsection{Results}
Summarize the main results. Use figures/tables for clarity:
\begin{itemize}
  \item Cluster quality metrics (e.g., silhouette score)
  \item Feature importances
  \item Visualizations of clusters or decision boundaries
\end{itemize}

\subsection{Discussion}
Interpret the results:
\begin{itemize}
  \item What patterns or groups emerged?
  \item Were the methods appropriate?
  \item Any limitations or anomalies?
\end{itemize}

% ------------------------------------------------
\section{Conclusion and Reflection}
Summarize what you found and learned:
\begin{itemize}
  \item Key insights from data mining
  \item Challenges faced and how you overcame them
  \item Potential future work or improvements
\end{itemize}

% ------------------------------------------------
\section*{Acknowledgements}
(Optional) Acknowledge any data sources, collaborators, or funding.

% ------------------------------------------------
\bibliographystyle{abbrv}
\bibliography{references}

\end{document}
